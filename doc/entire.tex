\part*{全体を通して}
能代宇宙イベントからの改善点として以下を挙げる.
\begin{itemize}
\item メンバー参加率の向上\newline
能代宇宙イベントでは多くの作業が一部のメンバーに集中していた.今回は仕事の分担を更に行って一人あたりの作業量を軽減した.
\item 分業の進行\newline
今回の大会を経て設計,加工,コーディングなど各々のメンバーが担当する分野が明確になり,専門性が高まった.
\end{itemize}
一方で全体を通して以下のような課題が残った
\begin{itemize}
\item スケジュールの遅延\newline
スケジュール管理はガントチャート及び定例の会議で行っていたが,スケジュール通りに進捗が進行せず,
会議ではその場その場でメンバーに仕事を振るような形になった.
\item 実験の不足\newline
上のスケジュール遅延も重なり,スケジュール通りに行えた実験は投下実験のみだった.
大会は中止となったが,大会時点では走行試験,落下試験しかできておらずE2E試験などを実施する必要がある.
\end{itemize}
\part*{今後について}
今回製作した機体は次回の能代宇宙イベントに持ち込むため,今後は実験などを行いながら更に完成度を高めていく.
可能であれば次回の能代宇宙イベントには新入生から成るチームを加えて編成し,2台の機体を製作したいが,
現在新歓活動やサークル活動が大幅に制限されており現状では困難である.
\part*{謝辞}
この機体の製作では白坂成功先生の全面的な支援を賜り,厚く感謝を申し上げます.
