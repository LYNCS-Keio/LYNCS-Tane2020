\documentclass[uplatex,a4j,11pt]{jsarticle}


% 数式
\usepackage{amsmath,amsfonts}
\usepackage{bm}
% 画像
\usepackage[dvipdfmx]{graphicx}

\usepackage{multicol}
\usepackage{siunitx}

\usepackage{url}

\makeatletter
\def\fgcaption{\def\@captype{figure}\caption}
\makeatother
\newcommand{\mfig}[3][width=15cm]{
\begin{center}
    \includegraphics[#1]{#2}
\fgcaption{#3 \label{fig:#2}}
\end{center}
}
%https://blog.kassyi.com/2012/10/latex-multicols-figtuer-and-caption.html
%画像挿入方法は上記リンクを参照

\begin{document}

\title{種子島ロケットコンテスト参加報告}
\author{宇宙科学総合研究会 LYNCS 佐々木良輔}
\date{\today}

%\columnseprule=0.1mm
\maketitle
\begin{abstract}
  2020年度の種子島ロケットコンテストは1年生のみ6人のチームで参加予定だった.
  本イベントに向けて第15回能代宇宙イベントに参加した機体の改良版を製作したが,
  COVID-19の感染拡大を受け大会が中止となったため,ここでは前回の機体からの改良点などを主に書きたいと思う.
\end{abstract}
\begin{multicols}{2}
\part*{機体}
\section{車輪}
前回の機体では車輪全体を3Dプリンターで造形していたため,落下衝撃に弱く,
車輪の製造に10時間以上の時間がかかり,また車輪径が小さく走破力の低いものだった.
以上を踏まえ今回の機体では 
\begin{itemize}
  \item 落下衝撃に強く
  \item 製造時間が短く
  \item 走破力が高い
\end{itemize}
という点を考慮して車輪を設計し,結果以下の2種類の車輪を製作した.
\vfill\null
\columnbreak
\subsection{板バネ\&ゴム車輪}
この車輪は以下の部品で構成される.
\begin{itemize}
  \item ホイール
  \item ハブ
  \item スポーク
  \item 板ばね
\end{itemize}
ホイール,ハブは3Dプリンタで製作している.スポークは一般的な輪ゴム,板バネは0.8mm厚のステンレス板から成る.
radial方向の荷重をスポーク,\ axial方向の荷重を板バネで吸収する.

3Dプリンタ製の部分を大幅に減らしたことで,製作に必要な時間は5時間程度となった.

しかし大会が中止となったことで実際の走破力を見ることはできなかった.
また板バネの降伏荷重が低くaxial方向の荷重に弱いため,板バネの強化などの要改善点が見つかった.
\vfill\null
\columnbreak
\subsection{スポンジ車輪}
この車輪はホイールからハブまでの全体がEVA(Ethylene-Vinyl Acetate)スポンジから成る.
30分程度で製作可能であり,\ radial,\ axial両方向の荷重に対して十分な柔軟性を持つ。

板バネ\&ゴム車輪と同様に走破力の評価は行えていないが,車輪面の凹凸が無いため走破力はいくらか低いことが予想される.
そのため,今後はウォータージェット加工機などを用いて凹凸がある走破力の高いスポンジ車輪の開発を行う.
\section{パラシュート分離機構}
\section{筐体}
\vfill\null
\columnbreak
\part*{電装}
\section{設計}
図\ref{fig:electric/block.jpg}に電装のブロック図を示す.新しい電装の特徴は以下の通りである.
\begin{itemize}
\item TWE-LITEを用いたアップ、ダウンリンク
\item 高精度気圧センサーDPS310の採用
\item 電源監視ICの搭載
\item 大電流駆動用のMOSFET
\end{itemize}
TWE-LITEを搭載したことで地上へのテレメトリ送信や,遠隔操作が可能になった.
また,前回の課題であった気圧センサーの精度を解決するため,新しくDPS310センサーを搭載した.
このセンサーでは標準偏差5\si{\cm}での高度測定が可能である.
電源監視ICを搭載したことで,電池残量や消費電力をリアルタイム監視が可能になり,バッテリーの交換タイミングを把握できる.
また,バッテリーからの電流を直接ドライブするMOSFETを搭載し,これはパラシュート分離機構に用いられている.
\mfig[width=7cm]{electric/block.jpg}{ブロック図}
\section{実装}
図\ref{fig:electric/elec1.jpg}は実際に製作した回路である.すべての部品はSMDで実装されており,高密度かつ耐衝撃性の高い電装を目指した.
\mfig[width=5cm]{electric/elec1.jpg}{実際の電装}
\section{問題点}
実際に製作するとGNSSを受信しないという問題が発生した.他の部品と切り分けて受信実験などをした結果,原因として
raspberry pi zeroのクロック(1\si{\GHz})とGNSS受信波(1.1-1.5\si{\GHz})が干渉している可能性が疑われた.
対策として図\ref{fig:electric/elec3.jpg},図\ref{fig:electric/elec2.jpg}のような銅テープによる高周波シールド,スペーサーによるraspberry pi zeroと基板の遠隔化の両方を行ったところ正常に受信した.
\mfig[width=7cm]{electric/elec3.jpg}{高周波シールド}
\mfig[width=7cm]{electric/elec2.jpg}{スペーサー}

しかし,スペーサーを入れると基板の全高が高くなるため,スペーサーなしで正常に受信ができるように改良する必要がある.
また,受信不良の原因が本当にクロックとの干渉のみであるのかを検証することが現状では困難であり,様々な実験を行う必要がある.
\part*{ソフトウェア}
ソフトウェアは以下のような改善を行った.
\begin{itemize}
  \item 例外設計
  \item configファイルによる設定の分離
  \item コメントアウトの充実
  \item 地磁気を併用した誘導
\end{itemize}
前回の機体ではGNSSを2点測位することで自己の方位を推定していたが,地磁気を用いることで高速かつ高精度に方位を推定することができた.
また,例外設計を行いエラーに対する耐性を高めた.

一方で以下のような要改善点,展望も残っている.
\begin{itemize}
  \item pythonからc++への変更による高速化
  \item 画像処理へのGPGPUの利用
  \item TWE-LITE送受信用データ形式の決定
  \item 地上局ソフトウェアの開発
\end{itemize}
また,実践的な実験が行える環境が少なく,実際の環境を想定して更に課題を抽出する必要がある.

\end{multicols}

\end{document}