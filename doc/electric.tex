\part*{電装}
\section{設計}
図\ref{fig:electric/block.jpg}に電装のブロック図を示す.新しい電装の特徴は以下の通りである.
\begin{itemize}
\item TWE-LITEを用いたアップ、ダウンリンク
\item 高精度気圧センサーDPS310の採用
\item 電源監視ICの搭載
\item 大電流駆動用のMOSFET
\end{itemize}
TWE-LITEを搭載したことで地上へのテレメトリ送信や,遠隔操作が可能になった.
また,前回の課題であった気圧センサーの精度を解決するため,新しくDPS310センサーを搭載した.
このセンサーでは標準偏差5\si{\cm}での高度測定が可能である.
電源監視ICを搭載したことで,電池残量や消費電力をリアルタイム監視が可能になり,バッテリーの交換タイミングを把握できる.
また,バッテリーからの電流を直接ドライブするMOSFETを搭載し,これはパラシュート分離機構に用いられている.
\mfig[width=7cm]{electric/block.jpg}{ブロック図}
\section{実装}
図\ref{fig:electric/elec1.jpg}は実際に製作した回路である.すべての部品はSMDで実装されており,高密度かつ耐衝撃性の高い電装を目指した.
\mfig[width=5cm]{electric/elec1.jpg}{実際の電装}
\section{問題点}
実際に製作するとGNSSを受信しないという問題が発生した.他の部品と切り分けて受信実験などをした結果,原因として
raspberry pi zeroのクロック(1\si{\GHz})とGNSS受信波(1.1-1.5\si{\GHz})が干渉している可能性が疑われた.
対策として図\ref{fig:electric/elec3.jpg},図\ref{fig:electric/elec2.jpg}のような銅テープによる高周波シールド,スペーサーによるraspberry pi zeroと基板の遠隔化の両方を行ったところ正常に受信した.
\mfig[width=7cm]{electric/elec3.jpg}{高周波シールド}
\mfig[width=7cm]{electric/elec2.jpg}{スペーサー}

しかし,スペーサーを入れると基板の全高が高くなるため,スペーサーなしで正常に受信ができるように改良する必要がある.
また,受信不良の原因が本当にクロックとの干渉のみであるのかを検証することが現状では困難であり,様々な実験を行う必要がある.