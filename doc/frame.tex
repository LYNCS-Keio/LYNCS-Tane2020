\part*{機体}
\mfig[width=7cm]{frame/20200417_192014.jpg}{機体全体図}
\section{車輪}
前回の機体では車輪全体を3Dプリンターで造形していたため,落下衝撃に弱く,
車輪の製造に10時間以上の時間がかかり,また車輪径が小さく走破力の低いものだった.
以上の問題を踏まえ
\begin{itemize}
  \item 落下衝撃に強く
  \item 製造時間が短く
  \item 走破力が高い
\end{itemize}
という点を考慮して車輪を設計し,結果以下の2種類の車輪を製作した.
\subsection{板バネ\&ゴム車輪}
この車輪は以下の部品で構成される.
\begin{itemize}
  \item ホイール
  \item ハブ
  \item スポーク
  \item 板ばね
\end{itemize}
ホイール,ハブは3Dプリンタで製作している.スポークは一般的な輪ゴム,板バネは0.8mm厚のステンレス板から成る.
radial方向の荷重をスポーク,\ axial方向の荷重を板バネで吸収する.

3Dプリンタ製の部分を大幅に減らしたことで,製作に必要な時間は5時間程度となった.

しかし大会が中止となったことで実際の走破力を見ることはできなかった.
また,板バネの降伏荷重が低くaxial方向の荷重に弱いため,板バネの強化などの要改善点が見つかった.
\vfill\null
\columnbreak
\mfig[width=6cm]{frame/Picture4.jpg}{板バネ\&ゴム車輪概観}
\subsection{スポンジ車輪}
この車輪はホイールからハブまでの全体がEVA(Ethylene-Vinyl Acetate)スポンジから成る.
30分程度で製作可能であり,\ radial,\ axial両方向の荷重に対して十分な柔軟性を持つ.

板バネ\&ゴム車輪と同様に走破力の評価は行えていないが,車輪面の凹凸が無いため走破力はいくらか低いことが予想される.
そのため,今後はウォータージェット加工機などを用いて凹凸がある走破力の高いスポンジ車輪の開発を行う.
\mfig[width=5cm]{frame/Picture6.jpg}{スポンジ車輪概観}
\vfill\null
\columnbreak
\section{パラシュート分離機構}
従来の電熱線式のパラシュート分離機構は
\begin{itemize}
  \item 火災の危険性
  \item 電熱線巻きの不良による動作不良
  \item 再使用時に巻き直しが必要
  \item 消費電力が大きい
\end{itemize}
などの問題があった.これを改善するために,前回の機体ではサーボによる分離機構を考案したが
\begin{itemize}
  \item 待機電力の多さ
  \item ヨー方向荷重によるサーボの破損
\end{itemize}
などの問題が明らかになった.

以上の問題を踏まえ今回の機体では
\begin{itemize}
  \item 動作の信頼性が高い
  \item 再使用が容易
  \item 待機電力が低い
  \item 全方向からの荷重に対する耐性
\end{itemize}
をコンセプトに新たなパラシュート分離機構を開発した.

パラシュート分離機構はデカプラーとデカプラーレセプタクルの2つのコンポーネントから成る.
\mfig[width=7cm]{frame/Picture1.jpg}{パラシュート分離機構(左:デカプラーレセプタクル;右:デカプラー)}
\newpage
\mfig[width=6cm]{frame/Picture7.jpg}{パラシュート分離機構の開放動作(断面)}
図\ref{fig:frame/Picture7.jpg}のようにデカプラーのロックピンがレセプタクルのロック溝に掛かることで荷重を受け,回転動作によりロック・開放を切り替える.
図\ref{fig:frame/Picture3.jpg}にロック時のパラシュート分離機構の断面図を示す.
図\ref{fig:frame/Picture1.jpg}に示すようにデカプラーレセプタクルにはトーションばねとソレノイドが内蔵されている.
トーションばねがデカプラーのキー溝に引っかかることでデカプラーは常にロック解除方向の回転力を受ける.
デカプラーを一度ロック位置まで回転させると,キー穴にソレノイドのピストンが掛かり,デカプラーは完全にロックされる.
ソレノイドはプル型ソレノイドであり,通電されるまでネガティブに機構をロックし続ける.
また,ディテクタスイッチを搭載したことでパラシュート分離の成否を検出できる.
\mfig[width=7cm]{frame/Picture3.jpg}{パラシュート分離機構(ロック時;断面)}
この機構はロック動作がネガティブであり,更にアクチュエーターをサーボからソレノイドに変更したことで動作の信頼性を向上し,
待機電力は0,動作時電力も3\si{\watt}未満を実現した.また,デカプラーを手で押し込むだけでロック状態になるため再使用も非常に容易である.
また,各方向の荷重を機械的に受けているため,前回の機体のようなアクチュエーター破損の可能性は大幅に削減できた.
しかし,\ 3Dプリンター製の部品の強度算出方法が不明であり,今後その評価方法を検討したい.

さらに意図せず発生したこの機構の特徴として,応答速度の早さが挙げられる.スローモーション動画\footnote{\url{https://drive.google.com/file/d/1Y7MWOB4V_zlWzZBSjIcJfnIA4Vlu7jlU/view?usp=sharing}}
(960fps)を用いて通電時のスパークから,デカプラー開放状態になるまでの時間を測定したところ0.031\si{\sec}以下という結果が得られた.
\newpage
\section{筐体}
前回の機体では筐体のCFRPプレートに鋭利な内側角があり,パラシュートの開傘衝撃で図\ref{fig:frame/Picture2.jpg}の赤線のように亀裂が入った.
そのため今回の機体では図\ref{fig:frame/Picture5.jpg}のように角を取り,
またパラシュート分離機構をアルミ製L字アングルで取り付けることでこれを解消した.
\mfig[width=7cm]{frame/Picture2.jpg}{前機体のCFRPプレート}
\mfig[width=6cm]{frame/Picture5.jpg}{新機体のCFRPプレート}