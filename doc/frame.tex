\part*{機体}
\section{車輪}
前回の機体では車輪全体を3Dプリンターで造形していたため,落下衝撃に弱く,
車輪の製造に10時間以上の時間がかかり,また車輪径が小さく走破力の低いものだった.
以上を踏まえ今回の機体では 
\begin{itemize}
  \item 落下衝撃に強く
  \item 製造時間が短く
  \item 走破力が高い
\end{itemize}
という点を考慮して車輪を設計し,結果以下の2種類の車輪を製作した.
\vfill\null
\columnbreak
\subsection{板バネ\&ゴム車輪}
この車輪は以下の部品で構成される.
\begin{itemize}
  \item ホイール
  \item ハブ
  \item スポーク
  \item 板ばね
\end{itemize}
ホイール,ハブは3Dプリンタで製作している.スポークは一般的な輪ゴム,板バネは0.8mm厚のステンレス板から成る.
radial方向の荷重をスポーク,\ axial方向の荷重を板バネで吸収する.

3Dプリンタ製の部分を大幅に減らしたことで,製作に必要な時間は5時間程度となった.

しかし大会が中止となったことで実際の走破力を見ることはできなかった.
また板バネの降伏荷重が低くaxial方向の荷重に弱いため,板バネの強化などの要改善点が見つかった.
\vfill\null
\columnbreak
\subsection{スポンジ車輪}
この車輪はホイールからハブまでの全体がEVA(Ethylene-Vinyl Acetate)スポンジから成る.
30分程度で製作可能であり,\ radial,\ axial両方向の荷重に対して十分な柔軟性を持つ。

板バネ\&ゴム車輪と同様に走破力の評価は行えていないが,車輪面の凹凸が無いため走破力はいくらか低いことが予想される.
そのため,今後はウォータージェット加工機などを用いて凹凸がある走破力の高いスポンジ車輪の開発を行う.
\section{パラシュート分離機構}
\section{筐体}